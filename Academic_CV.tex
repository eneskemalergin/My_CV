\documentclass[margin,line]{res}


\usepackage{hyperref}
\hypersetup{
   colorlinks,
   citecolor=black,
   filecolor=black,
   linkcolor=black,
   urlcolor=blue
}

\usepackage{url}

\oddsidemargin -.5in
\evensidemargin -.5in
\textwidth=6.0in
\itemsep=0in
\parsep=0in

%if using pdflatex:
\setlength{\pdfpagewidth}{\paperwidth}
\setlength{\pdfpageheight}{\paperheight}

\newenvironment{list1}{
  \begin{list}{\ding{113}}{%
      \setlength{\itemsep}{0in}
      \setlength{\parsep}{0in} \setlength{\parskip}{0in}
      \setlength{\topsep}{0in} \setlength{\partopsep}{0in}
      \setlength{\leftmargin}{0.17in}}}{\end{list}}
\newenvironment{list2}{
  \begin{list}{$\bullet$}{%
      \setlength{\itemsep}{0in}
      \setlength{\parsep}{0in} \setlength{\parskip}{0in}
      \setlength{\topsep}{0in} \setlength{\partopsep}{0in}
      \setlength{\leftmargin}{0.2in}}}{\end{list}}


\begin{document}

\name{Enes Kemal Ergin \vspace*{.1in}}

\begin{resume}
\section{\sc Contact Information}
\vspace{.05in}
\begin{tabular}{@{}p{2.3in}p{3.5in}}
{\it Cell:} 778-751-1649            & \hfill {\it GitHub:}  \href{https://github.com/eneskemalergin}{eneskemalergin} \\
{\it Work:} 604-875-2000 ext:7146   & \hfill {\it E-mail:}  \href{mailto:eneskemalergin@gmail.com}{eneskemalergin@gmail.com}\\
                                    & \hfill {\it Portfolio:}  \href{eneskemalergin.github.io}{eneskemalergin.github.io} \\
\end{tabular}


\section{\sc Research Interests}
Computational biology, high-throughput omics data analysis, cancer proteomics, machine learning applications in biomedical research, systems biology, and bioinformatics.

\section{\sc Education}

% Uncomment for PhD
{\bf University of British Columbia (UBC)}, Vancouver, BC Canada\\
{\em Bioinformatics}
\begin{list1}
\item[]  PhD, Bioinformatics, 2019 - 2025
\begin{list2}
\vspace*{.05in}
\item {\bf Advisor:}  Philipp F. Lange
\item {\bf Thesis Title:}  ``Computational interrogation of proteoform dynamics in pediatric cancer''
\end{list2}
\end{list1}

{\bf University of British Columbia (UBC)}, Vancouver, BC Canada\\
{\em Bioinformatics}
\begin{list1}
\item[]  MSc, Bioinformatics, 2017 - 2019(Transfered)
\end{list1}

{\bf North American University (NAU)}, Houston, Texas USA\\
{\em Software Engineering}
\begin{list1}
\item[] BSc, Computer Science,  2013 - 2017
\end{list1}


\section{\sc Research Experience}

{\bf Graduate Research Assistant}

\vspace{-.3cm}
{\em BC Children's Hospital Research Institute} \hfill {\bf September, 2017 - present}\\
\vspace{-.3cm}
\begin{list2}
\item {\bf Research:} Implementing statistical and computational methods for exploring proteomic features in childhood cancers based on bottom-up proteomics approaches focused on proteoforms.
\begin{list2}
\item {\bf Aim 1:} To expand statistical explainability through equivalence testing.
\item {\bf Aim 2:} To enhance peptide-level identification and quantification in proteomics data.
\item {\bf Aim 3:} To identify quantitatively distinct proteoforms at the peptide level in Neuroblastoma.
\end{list2}
\end{list2}

{\bf Visiting Researcher}

\vspace{-.3cm}
{\em Harvard Medical School} \hfill {\bf November, 2015 - September, 2016}\\
\vspace{-.3cm}
\begin{list2}
\item Research: Predicting the determinants of alternative mRNA splicing
\begin{list2}
\item {\bf Aim:} To determine how histone modifications influence the alternative splicing.
\item Developed a {\bf deep convolutional neural network} to predict mRNA expression patterns from 11 different histone modification datasets (pilot in HeLa cells).

\end{list2}
\item Research: Predicting transcription factor binding sites across cell types
\begin{list2}
\item {\bf Aim:} To develop a general framework to predict transcription factor (TF) binding sites accross various cell types.
\item Developed a machine learning-based algorithm can utilize large datasets (approximately 900GB) for estimating TF binding sites.
\end{list2}
\end{list2}

{\bf Undergraduate Research Assistant}

\vspace{-.3cm}
{\em North American University} \hfill {\bf September, 2015 - May, 2017}\\
Leading student in Bioinformatics Lab at NAU. Mentored 4 junior and sophomore students.\\
Designed an open source bioinformatics curriculum with Open Source Society in GitHub.

{\em Yeditepe University} \hfill {\bf August, 2014 - May, 2015}\\
Worked as a genomic data scientist to utilized virtual docking software to determine the best possible inhibitor for specific molecule by mainly using NCBI PubMed database.
%\begin{list2}
%\item
%\end{list2}

{\em Texas Institute of Education and Research (TIBER)} \hfill {\bf September, 2014 - April, 2015}\\
Wet-lab experience: Worked on drug development and testing process. Was responsible for preparing agar solutions, bacteria cultures, and liquid/solid drug tests on those bacteria cultures.


\section{\sc Teaching Experience}
{\bf Teaching Assistant}

\vspace{-.3cm}
{\em North American University} \hfill {\bf September 2015 - May 2017}\\
Co-taught 4 undergraduate level courses for computer science department to over 60 students each semester. Prepared the lab and extra sessions on Git/GitHub, Rapid Python Programming, and Ipython/Jupyter. Shared responsibilities for preparing lectures, exams, homework assignments, and  grading.
\vspace*{.05in}
\begin{list2}
\item COMP 3317 Algorithms, Fall 2015, 2016.
\item COMP 3320 Programming Languates, Fall 2016, Spring 2017.
\item COMP 3322 Software Engineering, Fall 2016.
\item COMP 2415 Systems Programming, Spring 2017
\end{list2}


{\bf Mathematics and Computer Science Tutor}
\vspace{-.1cm}

{\em North American University} \hfill {\bf January - May 2017}\\
Helped and advised students about their school classes. Tutored in various subjects from college algebra to differential equations in Mathematics domain and from CS1 to Data Mining in Computer Science domain.

{\bf Instructor}
\vspace{-.1cm}

{\em North American University} \hfill {\bf February - April 2015}\\
Taught Basic Python programming course to 35 people including students, faculty, and staff of NAU, which was a first ever course taught solely by a student in NAU. (\href{https://github.com/NAU-Python-Class/Py101-Spring-15}{Link})


\section{\sc Publications}
{\bf Under Review}

\vspace*{-.3cm}
PROFYLE. Proteomics and personalized patient-derived xenograft models identify treatment opportunities for a progressive malignancy within a clinically actionable timeframe and change care

{\bf Published}

\vspace*{-.3cm}

{\bf Ergin, E. K.}, Myung, J. J. K., Lange, P. F. (2024). Statistical Testing for Protein Equivalence Identifies Core Functional Modules Conserved across 360 Cancer Cell Lines and Presents a General Approach to Investigating Biological Systems., \href{https://doi.org/10.1021/acs.jproteome.4c00131}{doi: 10.1021/acs.jproteome.4c00131} {\em Journal of Proteome Research}

Lorentzian, A. C., Rever, J., {\bf Ergin, E. K.}, Guo, M., Akella, N. M., Rolf, N., Lim, C. J., Reid, G. S. D., Maxwell, C. A., Lange, P. F. (2023). Targetable lesions and proteomes predict therapy sensitivity through disease evolution in pediatric acute lymphoblastic leukemia, \href{https://doi.org/10.1038/s41467-023-42701-9}{doi: 10.1038/s41467-023-42701-9}, {\em Nature Communications}

{\bf Ergin, E.K.}, Uzozie, A.C., Chen, S., Su, Y., Lange, P.F. (2022). SQuAPP - Simple Quantitative Analysis of Proteins and PTMs, \href{https://doi.org/10.1093/bioinformatics/btac628}{doi: 10.1093/bioinformatics/btac628}, {\em Bioinformatics}

Uzozie A., {\bf Ergin E.K}, Rolf N., Tsui J., Lorentzian A., Weng S.H,  Nierves L., Smith T., Lim C.J,  Maxwell C., Reid S.G, Lange P.F. (2022). PDX models reflect the proteome landscape of pediatric acute lymphoblastic leukemia but divert in select pathways, \href{https://dx.doi.org/10.1186%2Fs13046-021-01835-8}{doi: 10.1186/s13046-021-01835-8}, {\em Journal of Experimental and Clinical Cancer Research}

Weng S.H*, Demir F.*, {\bf Ergin E.K}, Dirnberger S., Uzozie A., Tuscher D., Nierves L., Tsui J., Huesgen P.F, Lange P.F. (2019). Sensitive determination of proteolytic proteoforms in limited microscale proteome samples, \href{https://doi.org/10.1074/mcp.TIR119.001560}{doi: 10.1074/mcp.TIR119.001560}, {\em Molecular and Cellular Proteomics}

Kocabas F., {\bf Ergin E.K}. (2016). Identification of small molecule binding pocket for inhibition of Crimean-Congo hemorrhagic fever virus OTU proteasem \href{https://www.researchgate.net/profile/Fatih_Kocabas/publication/284188739_Identification_of_small_molecule_binding_pocket_for_inhibition_of_Crimean-Congo_hemorrhagic_fever_virus_OTU_protease/links/564f6a0b08aefe619b11de98.pdf}{doi: 10.3906/biy-1501-56} {\em Turkish Journal of Biology}, .


\section{\sc Presentations}
{\bf Oral Presentations}

\vspace{-.3cm}
{\em Trainee Omics Group (TOG)} \hfill {\bf May 2020}\\
{\bf Title:} Normalization in Proteomics: Past, Present, and Future

{\em 2023 Canadian National Proteomics Network} \hfill {\bf May 2023}\\
{\bf Title:} InPACCT: Integrated Proteomics Analysis of Curated Childhood Tumours

{\em Trainee Omics Group (TOG)} \hfill {\bf June 2023}\\
{\bf Title:} Exploring Stable Proteome in Cancer Cell Line Atlas with QuEStVar

{\bf Workshop Presentations}

\vspace{-.3cm}
{\em Trainee Omics Group (TOG)} \hfill {\bf July 2023}\\
{\bf Title:} Proteomics Workshop - Common Downstream Analysis Pipeline

{\bf Poster Presentations}

\vspace{-.3cm}
Ji J., Thibodeau M., {\bf Ergin E.K.}, Culibrk L., Smith T.. (2018). SWI/SNF Chromatin Remodeling Complex in Clear Cell Ovarian Cancer. Stat540 Group Project Poster Session.

Rostin K., {\bf Ergin E.K.}, Lange P.F. (2023). Insightful, Integrated Analyses of Public Pediatric Cancer Proteomics with InPACCT, BIG 23 Research Day, Vancouver, BC, Canada.

{\bf Ergin E.K.}, Lange P.F. (2023). Stable Proteome in Cancer Cell Lines with Combined Testing, BIG 23 Research Day, Vancouver, BC, Canada.

{\bf Ergin E.K.}, Lange P.F. (2023). Stable Proteome in Cancer Cell Lines with Combined Testing, 2023 Canadian National Proteomics Network, Regina, SK, Canada.

{\bf Ergin E.K.}, Rostin K., Lange P.F. (2023). InPACCT: Integrated proteomics analyses of curated childhood tumours, 2023 Canadian National Proteomics Network, Regina, SK, Canada.

Yadegari M., {\bf Ergin E.K.}, Lange P.F. (2023) Detection of Shed Surface Proteins and Processed Oncoproteoforms in Biofluids, Canadian Cancer Research Conference 2023, Halifax, NS, Canada.

Jenane L., Niervez L., {\bf Ergin E.K.} Lange P.F. (2024) Characterization of a New Type of Neoantigen in T-Cell Acute Lymphoblastic Leukemia (T-ALL) by Cell Surface Terminomics, US HUPO 2024, Portlan OR, USA.


\section{\sc Projects}

% Ask this to people and fix descriptions.

{\bf Open Source Bioinformatics Curriculum} \hfill {\bf June  2016}\\
Developed a open source bioinformatics curriculum with contributions of open source society, which gives aspiring bioinformatics scientists chance to start building their foundational knowledge. (\href{https://github.com/open-source-society/bioinformatics}{Link})

{\bf Scholar Development Center} \hfill {\bf February  2016}\\
Created a non-profit community based organization under the Raindrop Foundation to help Turkish undergraduate students around Texas to achieve their dreams in academia or industry.

\vspace{\baselineskip}

\section{\sc Honors and Awards}

Four Year Doctoral Fellowship Award, 2019-2023

\vspace*{-2.5mm}

Michael Cuccione Childhood Cancer Foundation Graduate Studentship, 2018-2019

\vspace*{-2.5mm}
North American University: Exceptional Merit Scholarship, 2012-2017

\vspace*{-2.5mm}
North American University: President's Honor Roll , 2015-2017

\vspace*{-2.5mm}
North American University: Outstanding Student of the Year, 2015


\section{\sc Extracurricular Activities}
\begin{list2}
\item ISCB (International society of computational biology), {\em Member} \hfill {\bf August, 2016 - present}
\item ACM (Association for Computing Machinery), {\em Member} \hfill  {\bf February 2013 - present}
\item NAU Kazakh Student Association, {\em Club Advisor} \hfill {\bf September 2016 - May 2017}
\item NAU Future Leaders Club, {\em Founder and Director} \hfill {\bf April, 2015 - May 2017}
\item Student Government, {\em VP of Unity and Social Justice} \hfill  {\bf September 2014 - May 2017}
\item NAU ACM, {\em Member} \hfill {\bf September 2013 - May 2017}
\item NAU ACM, {\em Vice President} \hfill  {\bf September 2015 - May 2016}
\item NAU ACM, {\em Secretary} \hfill  {\bf February 2013 - September 2014}
\end{list2}

\section{\sc Technical Skills}
\begin{list2}
\item {\bf Languages:}  Python (Numpy, Pandas, Polars, Matplotlib, Seaborn, Scipy, Biopython, Scikit-learn, PyTorch, Tensorflow), R (tidyverse, shiny, limma, deseq2, msstats), Java, \LaTeX, C/C++, Shell/Bash Scripting, Javascript, HTML, CSS
\item {\bf Database Systems:} SQL, MySQL, MongoDB
\item {\bf Operating Systems:}  Unix/Linux, MacOS, Windows.
\end{list2}

% \section{\sc References}
% \begin{list2}
% \item Assistant Prof. Dr.Philipp Lange, Pathology Department, University of British Columbia, Vancouver, BC, Canada, \href{philipp.lange@ubc.ca}{philipp.lange@ubc.ca}
% \item Assistant Prof. Dr. Stirling L. Churchman, Genetics Department, Harvard Medical School, Boston, MA, USA, \href{churchman@genetics.med.harvard.edu}{churchman@genetics.med.harvard.edu}
% \item Associate Prof. Dr. Kemal Aydin, Computer Science Department, North American University, Houston, TX, USA, \href{kemal@na.edu}{kemal@na.edu}
% \item Prof. Dr. Cengiz Zubeyir Altuntas, Director of Texas Institute of Biotechnology Education and Research, North American University, Houston, TX, USA \href{cza@na.edu}{cza@na.edu}
% \end{list2}


\end{resume}
\end{document}
